
\chapter{Functions}
\label{functions}

\noindent Functions\index{function} are objects that can have arguments. 
There exist several types of functions in FORM. First there is the 
distinction between commuting\index{commuting} and 
noncommuting\index{noncommuting} functions. Commuting functions commute 
with all other objects. This property is used by the normalization routines 
that bring terms into standard form. Noncommuting functions do not commute 
necessarily with other noncommuting functions. They do however commute with 
objects that are considered to be commuting, like symbols, vectors and 
commuting functions. Various instances of the same noncommuting function 
but with different arguments do not commute either.

\noindent The next subdivision of the category of functions is in regular 
functions\index{function!regular}, tensors\index{tensor} and 
tables\index{table}. Tensors are special functions that can have only 
indices or vectors for their arguments. If an argument is a vector, it is 
assumed that this vector is there as the result of an index contraction. 
Tables are functions with automatic substitution rules. A table must have 
at least one table\index{table index} index\index{index!table}. Each time 
during normalization FORM will check whether an instance of a table can be 
substituted. This means that undefined table elements will slow the program 
down somewhat.

\noindent All the various types of functions are declared with their own 
declaration statements. These are described in the chapter for the 
statements (see chapter~\ref{statements}).

One of the useful properties of functions is the 
wildcarding\index{wildcard} of their arguments during pattern matching. The 
following argument wildcards are possible:

\leftvitem{2cm}{x?}
\rightvitem{14cm}{Here x is a symbol. This symbol can match either a 
symbol, any numerical argument, or a complete subexpression argument that 
is not vectorlike or indexlike.}

\leftvitem{2cm}{i?}
\rightvitem{14cm}{Here i is an index. This index can match either an index, 
a vector (actually the dummy\index{dummy} index\index{index!dummy} of the 
vector that was contracted), or a complete subexpression that is vector like 
(again actually the contracted dummy index).}

\leftvitem{2cm}{v?}
\rightvitem{14cm}{Here v is a vector. This vector can match either a vector 
or a complete subexpression that is vector like.}

\leftvitem{2cm}{f?}
\rightvitem{14cm}{Here f is any functiontype. This function can match any 
function. It is the responsibility of the user to avoid problems in the 
righthandside if f happens to match a tensor.}

\leftvitem{2cm}{?a}
\rightvitem{14cm}{This is an argument\index{argument field} field 
wildcard\index{wildcard!argument field}. This can match a 
complete set of arguments. The set can be empty. Argument field wildcards 
have a name that starts with a question mark followed by a name. They do 
not have to be declared as there cannot be confusion.}

\noindent In addition to the above syntax FORM knows a number of special 
functions with well defined properties. All these functions have a name 
that ends in an underscore. In addition the names of these built in objects 
are case insensitive. This means for instance that the factorial function 
can be refered to as \verb:fac_:, \verb:Fac_: or \verb:FAC_: or whatever 
the user considers more readable. The built in functions are:

\section{abs\_}\index{abs\_}\index{function!abs\_}
\label{funabs}
\noindent If one argument which is numerical it evaluates into the 
absolute value of the argument.

\section{bernoulli\_}\index{bernoulli\_}\index{function!bernoulli\_}
\label{funbernoulli}
\noindent If it has one nonzero integer argument n it evaluates into 
the n-th coefficient in the power series expansion of $x/(1-e^{-x})$.

\section{binom\_}\index{binom\_}\index{function!binom\_}
\label{funbinom}
\noindent binom\_(n,i) $= n!/(i!(n-i)!)$. If the arguments are non 
integer or negative, no substitution is made.

\section{conjg\_}\index{conjg\_}\index{function!conjg\_}
\label{funconjg}
\noindent Currently not doing anything.

\section{count\_}\index{count\_}\index{function!count\_}
\label{funcount}
\noindent Similar to the count object in the if statement (see 
\ref{substaif}). This function expects the same arguments as the count 
object and returns the corresponding count value for the current term.

\section{d\_}\index{d\_}\index{function!d\_}
\label{fund}
\noindent The kronecker\index{kronecker} delta\index{delta!kronecker}. 
Should have two indices for arguments. Often indicated as 
$\delta^{\mu\nu}$. In automatic summation over the indices the d\_ often 
vanishes again as in
\verb:d_(mu,nu)*p(mu)*q(nu): $\rightarrow$ \verb:p.q: and similar 
replacements. Internally this object is treated in a rather special way. 
Hence it will not match a function wildcard.

\section{dd\_}\index{dd\_}\index{function!dd\_}
\label{fundd}
\noindent This is a combinatorics\index{combinatorics} function. The tensor 
dd\_ with an even number of indices is equal to the totally symmetric 
tensor built up from products of kronecker delta's. Each term in this 
symmetric combination is normalized to one. In principle there are 
$n!/(2^{n/2}(n/2)!$ terms in this combination. The profit comes when some 
or all the indices are contracted with vectors and some of these vectors 
are identical. In that case FORM will use combinatorics to generate only 
different terms, each with the proper prefactor. This can result in great 
time and space savings.

\section{delta\_}\index{delta\_}\index{function!delta\_}
\label{fundelta}
\noindent If one argument and it is numerical the result is one if 
the argument is zero and zero otherwise. If two arguments, the result is 
one if the arguments are numerical and identical. If they are numerical and 
they differ the result is zero. In all other cases nothing is done.

\section{deltap\_}\index{deltap\_}\index{function!deltap\_}
\label{fundeltap}
\noindent If one argument and it is numerical the result is zero if 
the argument is zero and one otherwise. If two arguments, the result is 
zero if the arguments are numerical and identical. If they are numerical and 
they differ the result is one. In all other cases nothing is done.

\section{denom\_}\index{denom\_}\index{function!denom\_}
\label{fundenom}
\noindent Internal function to describe denominators. Has a single 
argument. \verb:den(a+b): is printed as \verb:1/(a+b):.

\section{distrib\_}\index{distrib\_}\index{function!distrib\_}
\label{fundistrib}
\noindent This is a combinatorics\index{combinatorics} function. It should 
have at least five arguments. If we have
\begin{verbatim}
    distrib_(type,n,f1,f2,x1,...,xm)
\end{verbatim}
with type and n integers, f1 and f2 functions and then a number of 
arguments there can be action if $-2 \le$ type $\le 2$. The typical action 
is that the arguments \verb:x1,...,xm: will be divided over the two 
functions in all possible ways. For each possibility a new term is 
generated. The relative order of the arguments is kept. If type is negative 
it is assumed that the collection of x-arguments is 
antisymmetric\index{antisymmetric} and hence the number of permutations 
needed to make the split will determine whether there will be a minus sign 
on the resulting term. When type is zero all possible divisions are 
generated. Hence there will be $2^m$ divisions. The second argument is then 
not relevant. If type is 1 or -1 the second parameter says that the first 
function should obtain n arguments. The remaining arguments go to the 
second function. If type is 2 or -2 the second function should obtain n 
arguments. Example:
\begin{verbatim}
    Symbols x1,...,x4;
    CFunctions f,f1,f2;
    Local F = f(x1,...,x4);
    id  f(?a) = distrib_(-1,2,f1,f2,?a);
    Print +s;
    .end

   F =
       + f1(x1,x2)*f2(x3,x4)
       - f1(x1,x3)*f2(x2,x4)
       + f1(x1,x4)*f2(x2,x3)
       + f1(x2,x3)*f2(x1,x4)
       - f1(x2,x4)*f2(x1,x3)
       + f1(x3,x4)*f2(x1,x2)
      ;
\end{verbatim}
When adjacent x-arguments are identical FORM uses combinatorics to avoid 
generating more terms than necessary.


\section{dum\_}\index{dum\_}\index{function!dum\_}
\label{fundum}
\noindent Special function for printing virtual\index{virtual bracket} 
brackets\index{bracket}. \verb:dum_(a+b): is printed as \verb:(a+b):: the 
name of this function is not printed!

\section{dummy\_}\index{dummy\_}\index{function!dummy\_}
\label{fundummy}
\noindent For internal use only.

\section{dummyten\_}\index{dummyten\_}\index{function!dummyten\_}
\label{fundummyten}
\noindent For internal use only.

\section{e\_}\index{e\_}\index{function!e\_}
\label{fune}
\noindent The Levi-Civita\index{Levi-Civita tensor} 
tensor\index{tensor!Levi-Civita}. It is a totally 
antisymmetric\index{antisymmetric} tensor with well defined contraction 
rules (see \ref{substacontract}).

\section{exp\_}\index{exp\_}\index{function!exp\_}
\label{funexp}
\noindent Internal function with two arguments. Represents 
argument1 to the power argument2. Of course it is printed in the standard 
power notation.

\section{fac\_}\index{fac\_}\index{function!fac\_}
\label{funfac}
\noindent The factorial\index{factorial} function. If it has a single nonzero 
integer argument n it is replaced by n! but if the result is bigger than 
the maximum allowable number an error will result.
 
\section{factorin\_}\index{factorin\_}\index{function!factorin\_}
\label{funfactorin}
\noindent When the argument is a single \$-variable\index{\$-variable} or 
an expression\index{expression} the function is replaced by the common 
factor in the terms of that \verb:$:-variable or expression. This common 
factor consists in the first place of all symbolic objects that occur in 
all terms. In addition the numerical factor consists of the QCD\index{QCD} 
of all numerators and the LCM\index{LCM} of all denominators. Hence if the 
\verb:$:-variable or expression is divided by the result of factorin\_ all 
coefficients become integer.

\section{firstbracket\_}\index{firstbracket\_}\index{function!firstbracket\_}
\label{funfirstbracket}
\noindent In the case that there is a single argument and this 
single argument is the name of an expression, this function is replaced by 
the part that is outside brackets in the first term of the expression. If 
there are no brackets the function is replaced by one.

\section{g5\_}\index{g5\_}\index{function!g5\_}
\label{fungfive}
\noindent The $\gamma_5$ Dirac gamma matrix. We assume here that it 
anticommutes with the other Dirac\index{Dirac} gamma\index{gamma matrices} 
matrices. Anybody who does not like that should program private libraries 
(this should not be too difficult with the cycle symmetric functions 
(see \ref{substafunctions})). There should be a single index to indicate 
the spinline.

\section{g6\_}\index{g6\_}\index{function!g6\_}
\label{fungsix}
\noindent There should be a single index to indicate the spinline. 
As in Schoonschip\index{Schoonschip} we use $\gamma_6 = 1+\gamma_5$.

\section{g7\_}\index{g7\_}\index{function!g7\_}
\label{fungseven}
\noindent There should be a single index to indicate the spinline. 
As in Schoonschip\index{Schoonschip} we use $\gamma_7 = 1-\gamma_5$.

\section{g\_}\index{g\_}\index{function!g\_}
\label{fung}
\noindent The Dirac\index{Dirac} gamma\index{gamma matrices} matrix. Its 
first argument should be an index (either symbolic or numeric). Then follow 
zero, one or more indices to indicate a string of gamma matrices that 
belong together. Gamma matrices with the same first index are considered to 
belong together, but as long as the indices are symbolic no assumptions are 
made about whether they go together or not. Hence no commutation or 
anticommutation properties are applied for different spin lines unless the 
spinline indices are both numeric.

\section{gcd\_}\index{gcd\_}\index{function!gcd\_}
\label{fungcd}
%\noindent \verb:gcd_(x1,x2): is replaced by the greatest common 
%divisor of the two integers \verb:x1: and \verb:x2:. If the arguments are 
%not integers, nothing is done.
\noindent \verb:gcd_(x1,...,xn): is replaced by the greatest common divisor 
of the arguments. The arguments can be any combination of numbers and/or 
symbols. Any other objects will be a run time error. The arguments can be 
polynomials in which case the routines for `simple' polynomials are used. 
These routines put intermediate results in the workspace. The gcd of two 
polynomials is normalized such that the lead term has coefficient one. If 
however there is also an argument that is just a number the answer will be 
a number that is the gcd of all numbers in all arguments. When fractions 
are involved, the gcd is defined as the gcd of the numerators, divided by 
the gcd of the denominators.

\section{gi\_}\index{gi\_}\index{function!gi\_}
\label{fungi}
\noindent The unit Dirac gamma matrix. Should have a single index 
to indicate its spin line. Its is identical to a regular gamma matrix with 
no Lorenz indices: \verb:gi_(n) = g_(n):

\section{integer\_}\index{integer\_}\index{function!integer\_}
\label{funinteger}
\noindent This is a rounding\index{rounding} function. It should have 
either one or two arguments. If there is a single argument and it is 
numeric, it will be rounded down to become an integer. If there are two 
arguments of which the first is numeric and the second is either 1, 0 or 
-1, the result will be the rounded value of the first argument. If the 
second argument is 1, the rounding will be down, when it is -1, the 
rounding will be up and when it is zero the rounding will be towards zero. 
In all other cases nothing is done.

\section{invfac\_}\index{invfac\_}\index{function!invfac\_}
\label{funinvfac}
\noindent One divided by the factorial\index{factorial} function. If it has 
a single nonzero integer argument n it is replaced by 1/n! but if this 
results in a number bigger than the maximum allowable number an error will 
result.

\section{match\_}\index{match\_}\index{function!match\_}
\label{funmatch}
\noindent Currently not active. Replaced automatically by 1.

\section{max\_}\index{max\_}\index{function!max\_}
\label{funmax}
\noindent If all its arguments are numeric, this function returns 
the maximum value of these arguments.

\section{maxpowerof\_}\index{maxpowerof\_}\index{function!maxpowerof\_}
\label{funmaxpowerof}
\noindent If this function has a single argument that is a symbol, it 
returns the maximum power restriction of this symbol. If none was given it 
will be the installation dependent value MAXPOWER which is 10000 on 
32\index{32 bits} bit machines and 500000000 on 64\index{64 bits} bit 
machines.

\section{min\_}\index{min\_}\index{function!min\_}
\label{funmin}
\noindent If all its arguments are numeric, this function returns 
the minimum value of these arguments.

\section{minpowerof\_}\index{minpowerof\_}\index{function!minpowerof\_}
\label{funminpowerof}
\noindent If this function has a single argument that is a symbol, it 
returns the minimum power restriction of this symbol. If none was given it 
will be the installation dependent value -MAXPOWER which is -10000 on 32 bit 
machines.

\section{mod\_}\index{mod\_}\index{function!mod\_}
\label{funmod}
\noindent If there are two integer arguments and the second 
argument is a positive short integer (less than $2^{15}$ on 32 bit 
computers and less than $2^{31}$ on 64 bit computers) the return value is 
the first argument modulus the second. Note that if the second argument is 
not a prime number and the first argument contains a denominator, division 
by zero could occur. It is up to the user to avoid such cases.

\section{nargs\_}\index{nargs\_}\index{function!nargs\_}
\label{funnargs}
\noindent Is replaced by an integer indicating the number of 
arguments that the function has.

\section{nterms\_}\index{nterms\_}\index{function!nterms\_}
\label{funnterms}
\noindent If this function has only one argument it is replaced by 
the number of terms inside this argument.

\section{pattern\_}\index{pattern\_}\index{function!pattern\_}
\label{funpattern}
\noindent Currently not active. Replaced automatically by 1.

\section{poly\_}\index{poly\_}\index{function!poly\_}
\label{funpoly}
\noindent Experimental. No guarantees of anything. Not even speed. 
For internal use with the following polynomial functions.

\section{polyadd\_}\index{polyadd\_}\index{function!polyadd\_}
\label{funpolyadd}
\noindent Experimental. No guarantees of anything. Not even speed. The 
arguments should be two expressions\index{expression} or 
\$-variables\index{\$-variable} (one of each is also allowed). Both should 
contain only symbols with nonnegative integer powers. The result is the sum 
of the two polynomials.

\section{polydiv\_}\index{polydiv\_}\index{function!polydiv\_}
\label{funpolydiv}
\noindent Experimental. No guarantees of anything. Not even speed. 
The arguments should be two expressions\index{expression} or 
\$-variables\index{\$-variable} (one of each is 
also allowed). Both should contain only symbols with nonnegative integer 
powers. The result is the quotient in the polynomial division of the first 
polynomial divided by the second. Any remainder is discarded.

\section{polygcd\_}\index{polygcd\_}\index{function!polygcd\_}
\label{funpolygcd}
\noindent Experimental. No guarantees of anything. Not even speed. The 
arguments should be two expressions\index{expression} or 
\$-variables\index{\$-variable} (one of each is also allowed). Both should 
contain only symbols with nonnegative integer powers. The result is the 
Greatest Common Divisor of the two polynomials. Note that a: this function 
can be very slow if several variables are involved and no quick divisor is 
found. b: the speed is very sensitive to the order of declaration. c: the 
algorithm is far from optimal. The future should see great improvements 
here.

\section{polyintfac\_}\index{polyintfac\_}\index{function!polyintfac\_}
\label{funpolyfac}
\noindent Experimental. No guarantees of anything. Not even speed. The 
argument should be an expression\index{expression} or 
\$-variable\index{\$-variable}.
It should contain only symbols with nonnegative integer powers. The result 
is a polynomial that has been multiplied by a constant such that all 
coefficients are integer with content 1 (meaning that the GCD of all these 
coefficients is one).

\section{polymul\_}\index{polymul\_}\index{function!polymul\_}
\label{funpolymul}
\noindent Experimental. No guarantees of anything. Not even speed. The 
arguments should be two expressions\index{expression} or 
\$-variables\index{\$-variable} (one of each is also allowed). Both should 
contain only symbols with nonnegative integer powers. The result is the 
polynomial multiplication of both polynomials.

\section{polynorm\_}\index{polynorm\_}\index{function!polynorm\_}
\label{funpolynorm}
\noindent Experimental. No guarantees of anything. Not even speed. The 
argument should be one expression\index{expression} or 
\$-variable\index{\$-variable}.
It should contain only symbols with nonnegative integer powers. Normalizes 
the polynomial in such a way that the first term has coefficient one.

\section{polyrem\_}\index{polyrem\_}\index{function!polyrem\_}
\label{funpolyrem}
\noindent Experimental. No guarantees of anything. Not even speed. The 
arguments should be two expressions\index{expression} or 
\$-variables\index{\$-variable} (one of each is also allowed). Both should 
contain only symbols with nonnegative integer powers. The result is the 
remainder in the polynomial division of the first polynomial divided by the 
second polynomial.

\section{polysub\_}\index{polysub\_}\index{function!polysub\_}
\label{funpolysub}
\noindent Experimental. No guarantees of anything. Not even speed. The 
arguments should be two expressions\index{expression} or 
\$-variables\index{\$-variable} (one of each is also allowed). Both should 
contain only symbols with nonnegative integer powers. The result is the 
first polynomial minus the second polynomial.

\section{replace\_}\index{replace\_}\index{function!replace\_}
\label{funreplace}
\noindent This function defines a rather general purpose 
replacement\index{replacement} mechanism. It should have pairs of 
arguments. Each pair consists of a single symbol, index, vector or 
function, followed by what this object should be replaced by in the entire 
term. Functions can only be replaced by functions, indices only by indices. 
A vector can be replaced by a single vector or by a vector like expression. 
A symbol can be replaced by a single symbol, a numerical expression or a 
complete subexpression that is not index like or vector like. This 
mechanism is sometimes needed to make replacements in ways that are very 
hard with the id\index{id} statements because those do not make 
replacements automatically inside function arguments (see 
\ref{substaidnew}). It also allows to exchange two variables as the 
replacements are executed simultaneously by the wildcard substitution 
mechanism.
\begin{verbatim}
    Multiply replace_(x,y,y,x);
\end{verbatim}
will exchange x and y.

\section{reverse\_}\index{reverse\_}\index{function!reverse\_}
\label{funreverse}
\noindent Can only occur as an argument of a function. Is replaced 
by the reversed string of its own arguments.

\section{root\_}\index{root\_}\index{function!root\_}
\label{funroot}
\noindent If we have \verb:root_(n,x): and \verb:n: is a positive 
integer and \verb:x: is a rational number and \verb:y: is a rational number 
with $y^n = x$ (no imaginary numbers are considered and negative numbers 
are avoided if possible. Only one root is given) then \verb:root_(n,x): is 
replaced by \verb:y:. This function was originally intended for internal 
use. Do not hold it against the author that \verb:root_(2,1): is replaced 
by \verb:1:. In the case that it is needed the user should manipulate the 
sign or the complexity properties externally.

\section{setfun\_}\index{setfun\_}\index{function!setfun\_}
\label{funsetfun}
\noindent Currently not active.

\section{sig\_}\index{sig\_}\index{function!sig\_}
\label{funsig}
\noindent Is replaced by the sign of the (numerical) argument, i.e. by -1 
if there is a single negative argument and by +1 if there is a single 
numerical argument that is greater or equal to zero.

\section{sign\_}\index{sign\_}\index{function!sign\_}
\label{funsign}
\noindent \verb:sign_(n): is replaced by \verb:(-1)^n: if n is an 
integer.

\section{sum\_}\index{sum\_}\index{function!sum\_}
\label{funsum}
\noindent General purpose sum\index{sum} function. The first argument should 
be the summation parameter (a symbol). The second argument is the starting 
point of summation, the third argument the `upper' limit and a potential 
fourth argument the increment. These numbers should all be integers. 
Summation stops when the summation parameter obtains a value that has 
passed the upper limit. The last argument is the summant, the object to be 
summed over. It can be any subexpression. If it contains the summation 
parameter, it will be replaced by its value for each generated term. 
Examples:
\begin{verbatim}
    sum_(j,1,4,sign_(j)*x^j/j)
    sum_(i,1,9,2,sign_((i-1)/2)*x^i*invfac_(i))
\end{verbatim}

\section{sump\_}\index{sump\_}\index{function!sump\_}
\label{funsump}
\noindent Special sum function. Its arguments are like for the 
sum\_ function, but each new term is the product of the previously 
generated term with the last argument in which the current value of the 
summation parameter has been substituted. The first term is always one. 
Example:
\begin{verbatim}
    Symbol i,x;
    Local F = sump_(i,0,5,x/i);
    Print;
    .end

   F =
      1 + x + 1/2*x^2 + 1/6*x^3 + 1/24*x^4 + 1/120*x^5;
\end{verbatim}
This function is a leftover from the Schoonschip\index{Schoonschip} days. 
The ordinary sum\_ function is much more readable.

\section{table\_}\index{table\_}\index{function!table\_}
\label{funtable}
\noindent For action the arguments should be the name of a table and then 
either the name of a function or one symbol for each dimension of the 
table. In the case of the list of symbols the return value will be a 
monomial in the given symbols in which the powers of the symbols correspond 
to the table indices of the defined table elements with the coefficients 
the table contents corresponding to those indices. In the case of a 
function name the return value will be a sum over terms in which the table 
elements are indicated by arguments in the given function while these 
functions are then multiplied by the corresponding table elements. This is 
one way to put a complete table inside an expression and store it (with the 
save statement of \ref{substasave}) in a binary way for a future run in 
which the table can be filled again with the 
fillexpression\index{fillexpression} (see \ref{substafillexpression}) 
statement. Note that for obvious reasons one should avoid using symbols or 
functions that also occur inside the table definitions.

\section{tbl\_}\index{tbl\_}\index{function!tbl\_}
\label{funtbl}
\noindent This function is the `table stub function' as used by the 
tablebase\index{tablebase} construction. This is explained in chapter 
\ref{tablebase}. It is mainly for internal use, but it could occur in the 
output.

\section{term\_}\index{term\_}\index{function!term\_}
\label{funterm}
\noindent If the single argument is a \$-variable\index{\$-variable}, the 
term\_ function will be removed and the remainder of the current term will 
be used as value/contents for the \$-variable. This is one of the ways to 
put things inside a \$-variable.

\section{termsin\_}\index{termsin\_}\index{function!termsin\_}
\label{funtermsin}
\noindent If there is a single argument and this argument is the name of an 
active (or previously active during the current job) expression, the 
function is replaced by the number\index{number of terms} of terms in this 
expression. Stored expressions that were entered via a load statement (see 
\ref{substaload}) are excluded from this because for them FORM would have 
to actually count the terms.
 
\section{termsinbracket\_}\index{termsinbracket\_}\index{function!termsinbracket\_}
\label{funtermsinbracket}
\noindent If there is no argument, or the single argument is zero, the 
function is replaced by the number of terms in the current 
bracket\index{bracket}, provided the expression has been bracketted at its 
last sort and a keep brackets statement (see \ref{substakeep}) has been 
used. Note that the terms have to be counted. Hence this is a relatively 
expensive command. More options will be implemented in the future.

\section{theta\_}\index{theta\_}\index{function!theta\_}
\label{funtheta}
\noindent If there is a single numerical argument x the function is 
replaced by one if $x \ge 0$ and by zero if $x < 0$. If there are two 
numerical arguments $x_1$ and $x_2$ the function is replaced by one if $x_1 
= x_2$ or if the arguments are in natural order (if theta\_ would be a 
symmetric function there would be no reason to exchange the arguments) and 
by zero if the arguments are not in natural order (they would be exchanged 
in a symmetric function). In all other cases nothing is done.

\section{thetap\_}\index{thetap\_}\index{function!thetap\_}
\label{funthetap}
\noindent If there is a single numerical argument x the function is 
replaced by one if $x > 0$ and by zero if $x \le 0$. If there are two 
numerical arguments $x_1$ and $x_2$ the function is replaced by zero if $x_1 
= x_2$ or if the arguments are not in natural order. If the arguments are 
in natural order the function is replaced by one. In all other cases 
nothing is done.

\section{Extra reserved names}

\noindent In addition there are some names that have been reserved for 
future use. At the moment these functions do not do very much. It is hoped 
that in the future some simplifications of the arguments can be 
implemented. These functions are:

\leftvitem{3cm}{sqrt\_}\index{sqrt\_}\index{function!sqrt\_}
\rightvitem{13cm}{The regular sqare root.}

\leftvitem{3cm}{ln\_}\index{ln\_}\index{function!ln\_}
\rightvitem{13cm}{The natural logarithm.}

\leftvitem{3cm}{sin\_}\index{sin\_}\index{function!sin\_}
\rightvitem{13cm}{The sine function.}

\leftvitem{3cm}{cos\_}\index{cos\_}\index{function!cos\_}
\rightvitem{13cm}{The cosine function.}

\leftvitem{3cm}{tan\_}\index{tan\_}\index{function!tan\_}
\rightvitem{13cm}{The tangent function.}

\leftvitem{3cm}{asin\_}\index{asin\_}\index{function!asin\_}
\rightvitem{13cm}{The inverse of the sine function.}

\leftvitem{3cm}{acos\_}\index{acos\_}\index{function!acos\_}
\rightvitem{13cm}{The inverse of the cosine function.}

\leftvitem{3cm}{atan\_}\index{atan\_}\index{function!atan\_}
\rightvitem{13cm}{The inverse of the tangent function.}

\leftvitem{3cm}{atan2\_}\index{atan2\_}\index{function!atan2\_}
\rightvitem{13cm}{Another inverse of the tangent function.}

\leftvitem{3cm}{sinh\_}\index{sinh\_}\index{function!sinh\_}
\rightvitem{13cm}{The hyperbolic sine function.}

\leftvitem{3cm}{cosh\_}\index{cosh\_}\index{function!cosh\_}
\rightvitem{13cm}{The hyperbolic cosine function.}

\leftvitem{3cm}{tanh\_}\index{tanh\_}\index{function!tanh\_}
\rightvitem{13cm}{The hyperbolic tangent function.}

\leftvitem{3cm}{asinh\_}\index{asinh\_}\index{function!asinh\_}
\rightvitem{13cm}{The inverse of the hyperbolic sine function.}

\leftvitem{3cm}{acosh\_}\index{acosh\_}\index{function!acosh\_}
\rightvitem{13cm}{The inverse of the hyperbolic cosine function.}

\leftvitem{3cm}{atanh\_}\index{atanh\_}\index{function!atanh\_}
\rightvitem{13cm}{The inverse of the hyperbolic tangent function.}

\leftvitem{3cm}{li2\_}\index{li2\_}\index{function!li2\_}
\rightvitem{13cm}{The dilogarithm function.}

\leftvitem{3cm}{lin\_}\index{lin\_}\index{function!lin\_}
\rightvitem{13cm}{The polylogarithm function.}

\noindent The user is allowed to use these functions, but it could be that 
in the future they will develop a nontrivial behaviour. Hence caution is 
required.

